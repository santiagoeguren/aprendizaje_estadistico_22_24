%---------------------------------------------------------------------------------
%	PACKAGES
%----------------------------------------------------------------------------------

\documentclass[12pt]{article}

\usepackage[T1]{fontenc}
\usepackage[utf8]{inputenc}
%\usepackage[spanish]{babel}
\parindent=1cm %modificar tamaño de sangria 
\usepackage{amsmath}
\usepackage{amssymb,amsfonts,latexsym,cancel}
%-------------------------------------------------------------------------------------
%Librerias gráficas


\usepackage{graphicx}
\usepackage{epstopdf} %es para convertir imagenes eps a pdf es para que pueda compilar
\usepackage{float}
\usepackage{subfigure}


%--------------------------------------------------------------------------------------
%Librerias para vincular las citas

\usepackage{hyperref} 
\usepackage{xcolor}
\usepackage[export]{adjustbox}

\usepackage{footnote}

%Define el color de las biografia
\hypersetup{linkbordercolor=blue}



%####################################################################################
%Inicio Documento
%####################################################################################
\begin{document}





%####################################################################################
%	Carátula
%####################################################################################



\begin{titlepage}

\begin{center}
\vspace*{-1in}
\begin{figure}[htb]
\begin{center}
\includegraphics[width=6cm]{index}
\end{center}
\end{figure}








Universidad Champagnat\\
\vspace*{0.15in}

%Facultad de Ciencias Económicas \\
%\vspace*{0.6in}



\begin{Large}
\textbf{Statiscal learning and portfolio optimizacion under skewness probability distribution } \\
\end{Large}
\vspace*{0.3in}


%Técnicas de aprendizaje estadístico con asimetrías para la optimización de carteras de inversión

\rule{80mm}{0.1mm}\\
\vspace*{0.1in}


\begin{large}
Santiago Emiliano Eguren
\end{large}
\vspace*{0.3in}




\begin{large}
Rodri\\
\end{large}
\vspace*{0.3in}



\begin{large}
Cristian\\
\end{large}
\vspace*{0.3in}


\begin{large}
July 2022
\end{large}
\vspace*{0.3in}



\end{center}

\end{titlepage}



%Nueva Pagina
\newpage







%#####################################################################################
%	Seccion 
%#####################################################################################

\section{Introduction}

The aim of this paper is comprobate if is posible to optimize  effectively  portfolios under 
the assuption skewness probability distribution function.






%#####################################################################################
%	Seccion 
%#####################################################################################

\section{Basic Concepts}




%-------------------------------------------------------------------------------------
%Subsection
%-------------------------------------------------------------------------------------

\subsection{Multivariate Normal Variance Mixtures}


A random vector $\textbf{Z}=[Z_{1},Z_{2},...,Z_{n}]$ follows a normal variance mixture, if:


\begin{equation}
\textbf{Z}=\mu+\sqrt{W}AU
\label{eq_fdsfdsf4324}
\end{equation}\\


or

\begin{equation}
\begin{bmatrix}Z_{1}\\Z_{2}\\.\\.\\.\\Z_{n}\end{bmatrix}=
\begin{bmatrix}\mu_{1}\\ \mu_{2}\\.\\.\\.\\ \mu_{n}\end{bmatrix}
   \sqrt{W}
   \begin{bmatrix}
     a_{11} & a_{12}&.&.&.&a_{1n}\\
     a_{21} & a_{22}&.&.&.&a_{2n}\\
     . & .&.&.&.&.\\
     . & .&.&.&.&.\\
     . & .&.&.&.&.\\
     a_{n1} & a_{n2}&.&.&.&a_{nn}\\
   \end{bmatrix}
   U
\label{eq_fsdfsdf45rfdfse}   
\end{equation}\\


Where:\\


$\mathbf{\mu}$ $\in$ $\mathbb{R}^{n}$, is the location vector.\\

$W$ is a non-negative random variable independent of U.\\

$A$ $\in$ $\mathbb{R}^{nxn}$ denotes the scale matrix. Where $\Sigma=AA^{T}$ is the covarince matrix and $\Sigma$ $\in$ $\mathbb{R}^{nxn}$: \\



\begin{equation}
   \Sigma=
   \begin{bmatrix}
     Var[Z_{1}] & Cov[Z_{1},Z_{2}]&.&.&.&Cov[Z_{1},Z_{n}]\\
     Cov[Z_{2},Z_{1}] & Var[Z_{2}]&.&.&.&Cov[Z_{2},Z_{n}]\\
     . & .&.&.&.&.\\
     . & .&.&.&.&.\\
     . & .&.&.&.&.\\
     Cov[Z_{n},Z_{1}] & Cov[Z_{n},Z_{2}]&.&.&.&Var[Z_{n}]\\
   \end{bmatrix}   
\label{eq_fosdfhsdf8439e}   
\end{equation}\\


$\mathbf{U}$  $\in$ $\mathbb{R}^{n}$ denotes a standard normal random vector:

 
 
 
\begin{equation}
   U \sim N(
   \begin{bmatrix}0\\0\\.\\.\\.\\0\end{bmatrix},  
   \begin{bmatrix}
     1 & 0 &.&.&.&0\\
     0 & 1&.&.&.&0\\
     . & .&.&.&.&.\\
     . & .&.&.&.&.\\
     . & .&.&.&.&.\\
     0 & 0&.&.&.&1\\
   \end{bmatrix})   
\label{eq_fosdfijsdf8493}   
\end{equation}\\
 

%The Mixed Tempered Stable Paretian was introduced by Rroji and Mercuri \cite{Rroji_Mercuri} in $2014$. Is is a %generalization of the Normal Variance Mean Mixtures.  A ramdon variable variable $V$ is Mixed Tempered Stable
%distributed if:


%\begin{equation}
%V=\mu_{0} + \mu U +\sqrt{U} E
%\label{eq_MTStable}
%\end{equation}\\


%where:\\

%\begin{itemize}
%\item $\mu_{0}$, $\mu$ $\varepsilon$ $\Re$
%\item $V$ is a random variable defined in positive real number.???
%\item $E$ is a classical Tempered Stable random variable.???
%\end{itemize}


%-------------------------------------------------------------------------------------
%Subsection
%-------------------------------------------------------------------------------------

\subsection{Expected Shortfall}

Expected Shortfall is defined as the average loss beyond VaR:

\begin{equation}
ES_{\varepsilon}[Z_{t}]=E[-Z_{t} | -Z_{t} > VaR_{\varepsilon}[Z_{t}]]
\label{eq_sdfsdfdsf09f0ifmsfi9}
\end{equation}\\


ES is also know as Expected Tail Loss (ETL) or Conditional Value-at-Risk (CVAR). Usually $\varepsilon$ is equal to $0,01$ or $0,05$. ES is a convex function weights and hence is usefull to optimizate portfolios (see Rockafellar and Uryasev \cite{RockafellarUryasev}).






%#####################################################################################
%	Seccion 
%#####################################################################################

\section{Portfolio Optimizacion}

To optimize, we consider daily return data from the S\&P 500 index between $2008-01-01$ and
$2018-01-01$. We fit marginal ARMA(1, 1)-GARCH(1, 1) and then fit normal variance
mixture models to the resulting standardized residuals.

We consider the inverse-gamma distribution for $W$. The portfolio will contain 3 random stock.







%#####################################################################################
%	Seccion 
%#####################################################################################

\section{Conclusion}




%----------------------------------------------------------------------------------------
%	Bibliografía
%----------------------------------------------------------------------------------------
%Nueva Pagina
%\newpage
\clearpage

\begin{thebibliography}{10} %el 10 es el tamaño del [1]

	
%\bibitem{Artzner_Delbean} Artzner, P., Delbean, F., Eber, J. M. and Heath, D.,  \emph{Coherent measures of risk}, %Mathematical Finance 9, 203-228, 1999.

%\bibitem{Hitai_Friedrick_Mercuri_Rroji} Hitaj, Asmerilda, Hubalek, Friedrick, Mercurim  Lorenzo
%and Rroji, Edit\emph{On Properties of the MixedTS Distribution and its Multivariate Extension}, Article in %International Statiscal Review, May 2018	


\bibitem{RockafellarUryasev} Rockafellar, R.T. and Uryasev, S., \emph{Optimization of conditional value-at-risk},
Journal of Risk 3, 21-41,2000.








\end{thebibliography}

%#####################################################################################
%Tabla de contenidos
%#####################################################################################
%Nueva Pagina
\newpage
 


\tableofcontents



%####################################################################################
%Fin documento
%####################################################################################
\end{document}