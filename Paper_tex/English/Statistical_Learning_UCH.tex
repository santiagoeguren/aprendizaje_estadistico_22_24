%---------------------------------------------------------------------------------
%	PACKAGES
%----------------------------------------------------------------------------------

\documentclass[12pt]{article}

\usepackage[T1]{fontenc}
\usepackage[utf8]{inputenc}
%\usepackage[spanish]{babel}
\parindent=1cm %modificar tamaño de sangria 
\usepackage{amsmath}
\usepackage{amssymb,amsfonts,latexsym,cancel}
%-------------------------------------------------------------------------------------
%Librerias gráficas


\usepackage{graphicx}
\usepackage{epstopdf} %es para convertir imagenes eps a pdf es para que pueda compilar
\usepackage{float}
\usepackage{subfigure}


%--------------------------------------------------------------------------------------
%Librerias para vincular las citas

\usepackage{hyperref} 
\usepackage{xcolor}
\usepackage[export]{adjustbox}

\usepackage{footnote}

%Define el color de las biografia
\hypersetup{linkbordercolor=blue}



%####################################################################################
%Inicio Documento
%####################################################################################
\begin{document}





%####################################################################################
%	Carátula
%####################################################################################



\begin{titlepage}

\begin{center}
\vspace*{-1in}
\begin{figure}[htb]
\begin{center}
\includegraphics[width=6cm]{index}
\end{center}
\end{figure}








Universidad Champagnat\\
\vspace*{0.15in}

%Facultad de Ciencias Económicas \\
%\vspace*{0.6in}



\begin{Large}
\textbf{Statiscal learning and portfolio optimizacion } \\
\end{Large}
\vspace*{0.3in}


%Técnicas de aprendizaje estadístico con asimetrías para la optimización de carteras de inversión

\rule{80mm}{0.1mm}\\
\vspace*{0.1in}


\begin{large}
Santiago Emiliano Eguren
\end{large}
\vspace*{0.3in}




\begin{large}
Rodri\\
\end{large}
\vspace*{0.3in}



\begin{large}
Cristian\\
\end{large}
\vspace*{0.3in}


\begin{large}
July 2022
\end{large}
\vspace*{0.3in}



\end{center}

\end{titlepage}



%Nueva Pagina
\newpage







%#####################################################################################
%	Seccion 
%#####################################################################################

\section{Introduction}








%#####################################################################################
%	Seccion 
%#####################################################################################

\section{Basic Concepts}




%-------------------------------------------------------------------------------------
%Subsection
%-------------------------------------------------------------------------------------

\subsection{Mixed Tempered Stable Paretian}


The Mixed Tempered Stable Paretian was introduced by Rroji and Mercuri \cite{Rroji_Mercuri} in $2014$. Is is a generalization of the Normal Variance Mean Mixtures.  A ramdon variable variable $V$ is Mixed Tempered Stable
distributed if:


\begin{equation}
V=\mu_{0} + \mu U +\sqrt{U} E
\label{eq_MTStable}
\end{equation}\\


where:\\

\begin{itemize}
\item $\mu_{0}$, $\mu$ $\varepsilon$ $\Re$
\item $V$ is a random variable defined in positive real number.???
\item $E$ is a classical Tempered Stable random variable.???
\end{itemize}


%-------------------------------------------------------------------------------------
%Subsection
%-------------------------------------------------------------------------------------

\subsection{Risk Measure - ETL}

Expected Tail Loss is defined as the average loss beyond VaR:

\begin{equation}
ETL_{\varepsilon}[V]=E[-V | -V > VaR_{\varepsilon}[V]]
\label{eq_ETL_1}
\end{equation}\\


ETL is also know as Expected Shortfall (ES) or Conditional Value-at-Risk (CVAR). Usually $\varepsilon$ is equal to $0,01$ or $0,05$. ETL is a convex function weights and hence is usefull to optimizate portfolios (see Rockafellar and Uryasev \cite{RockafellarUryasev}).



\subsection{Ratio - STARI}

The Sharpe Ratio is the clasical portfolio performance measure, but it has a disadvantages. Martin, Svetlozar and Siboulet \cite{Martin_Rachev_Siboulet}  described this disadvantages as follow:




\begin{itemize}
\item The standard deviation is a symmetric measure that not focus on downside risk.
\item The standard deviation is not a coherent measure of risk (see Artzner \cite{Artzner_Delbean}).`
\item The estimate of standard deviation is a highly unstable when the portfolio has a heavy-tailed distribution.
\end{itemize}


This authors propose the \textit{Stable Tail Adjusted Return Indicator (STARI)} as an alternative performance measure that does not suffer these problems.  The STARI is defined as:
 
\begin{equation}
STARI_{\varepsilon}=\dfrac{E[ \Delta X]-r_{f}}{ETL_{\varepsilon}[V]}
\label{eq_STARI_e}
\end{equation}\\


Where:

\begin{itemize}
\item $E[\Delta X]$ is the expected return of portfolio.
\item $ETL_{\varepsilon}[V]$  is the Expected Tail Loss.
\item $r_{f}$ is the risk-free return
\end{itemize}




%#####################################################################################
%	Seccion 
%#####################################################################################

\section{Portfolio Optimizacion}





%-------------------------------------------------------------------------------------
\subsection{Portfolio without restriction???}



%-------------------------------------------------------------------------------------
\subsection{Portfolio with restriction???}




%#####################################################################################
%	Seccion 
%#####################################################################################

\section{Conclusion}




%----------------------------------------------------------------------------------------
%	Bibliografía
%----------------------------------------------------------------------------------------
%Nueva Pagina
%\newpage
\clearpage

\begin{thebibliography}{10} %el 10 es el tamaño del [1]

	
\bibitem{Artzner_Delbean} Artzner, P., Delbean, F., Eber, J. M. and Heath, D.,  \emph{Coherent measures of risk}, Mathematical Finance 9, 203-228, 1999.

\bibitem{Hitai_Friedrick_Mercuri_Rroji} Hitaj, Asmerilda, Hubalek, Friedrick, Mercurim  Lorenzo
and Rroji, Edit\emph{On Properties of the MixedTS Distribution and its Multivariate Extension}, Article in International Statiscal Review, May 2018	
	
\bibitem{Martin_Rachev_Siboulet} Martin R. Douglas, Svetlozar Rachev and Siboulet Frederic, \emph{Phi-alpha Optimal Portfolios and Extreme Risk }, Wilmott magazine, November 2003.


\bibitem{Rroji_Mercuri} Rroji, Edit and Mercuri, Lorenzo \emph{Mixed Tempered Stable Distribution}, Article in Quantitative Finance, October 2014.


\bibitem{RockafellarUryasev} Rockafellar, R.T. and Uryasev, S., \emph{Optimization of conditional value-at-risk},
Journal of Risk 3, 21-41,2000.








\end{thebibliography}

%#####################################################################################
%Tabla de contenidos
%#####################################################################################
%Nueva Pagina
\newpage
 


\tableofcontents



%####################################################################################
%Fin documento
%####################################################################################
\end{document}